\documentclass{article}

\usepackage[table]{xcolor}
\usepackage[paperwidth=800mm, paperheight=500mm]{geometry}

% Don't show page numbers.
\pagenumbering{gobble}

% Column settings for skills able.
\newcolumntype{C}{p{18em}}

\begin{document}

% Title of matrix.
\begin{center}
\begin{huge}
Systems Engineering
\end{huge}
\end{center}

\bigbreak

% Alternating row colors.
\rowcolors{2}{gray!80!gray!50}{gray!50!gray!20}

% Padding for each cell.
{\renewcommand{\arraystretch}{2}

% Skills table.
\begin{tabular}{|C|C|C|C|C|C|C|C|C|}
  \hline
  \rowcolor{blue!60!black!10}
    \textbf{Level/Skill}
    &
    \cellcolor{blue!25}
    \textbf{Product Output}
    &
    \textbf{Decision Making/Supervision}
    &
    \textbf{Communication/Writing}
    &
    \textbf{Data structures/Performance}
    &
    \textbf{Security}
    &
    \textbf{Networking}
    &
    \textbf{Systems Engineering}
    &
    \textbf{Tooling}
    \\
  \hline
    % Level
    1
    &

    % Product Output
    \cellcolor{blue!25}
    Creates a design document based on well-defined scoped requirements and
    implements it.
    &

    % Decision Making/Supervision
    Normally receives general instruction on work and new assignments.
    &

    % Communication/Writing
    Reports progress on a regular basis as required by the team's operational
    requirements. Actively solicits feedback. Participates on interview panels.
    &

    % Data structures/Performance
    Demonstrates good understanding of basic data structures like hash tables,
    linked lists, and trees. Can reason about algorithm complexity. Applies relevant
    data structures in day to day activity.

    \bigbreak

    Can implement a production quality software - it might be not the most
    efficient or secure, but correct.
    &

    % Security
    Applies basic security principles to program design. For example, can set
    up HTTPS and password based auth.
    &

    % Networking
    Understands and reasons about networking concepts. Understands and can
    write production quality web servers. Understands common networking issues and
    troubleshooting techniques.
    &

    % Systems Engineering
    Understands the usage of POSIX and other APIs for Linux systems.

    Understands synchronization primitives and their application, including
    reasoning about deadlocks and data races. Can write basic system-level code
    using the different types of memory and allocation. Understands inter process
    communication and can build systems leveraging it. Can implement data race and
    deadlock free code using basic production guidelines - using synchronization
    primitives and properly sharing state between components of the system.
    &

    % Tooling
    Understands the usage of compilers, interpreters, build tools at the organization.
    \\ [13em]
  \hline
    % Level
    2
    &

    % Product Output
    \cellcolor{blue!25}
    Can write high quality user and product focused documentation.
    &

    % Decision Making/Supervision
    Normally receives little instruction on day-to-day work, general instructions
    on new assignments.
    &

    % Communication/Writing
    Provides constructive review on peers' code and design. Helps new team
    members during their first weeks.
    &

    % Data structures/Performance
    &

    % Security
    Applies industry best practices and security guidelines, like setting up
    strong TLS, can pick strong authentication and authorization mechanisms.
    &

    % Networking
    Has more granular understanding of the networking design, for example can
    reason about using GRPC vs HTTPS-JSON and their networking and scalability
    trade-offs. Can do the same for UDP vs TCP and lower level protocols.
    &

    % Systems Engineering
    Can reason about performance implications and risks of using
    synchronization primitives, understands granularity of locking, can debug and
    troubleshoot memory and synchronization issues.
    &

    % Tooling
    Understands tools and compilers as an advanced user - for example, can
    refactor Makefiles to make them more efficient and parallel execution friendly
    or select the optimal set of compiler flags for Linux and macOS.
    \\ [13em]
  \hline
    % Level
    3
    &

    % Product Output
    \cellcolor{blue!25}
    Collaborates with the team to scope requirements, based on good
    understanding of existing longer term product vision and estimates of the
    system design of a feature of a product.
    &

    % Decision Making/Supervision
    Work is done independently and is reviewed at critical points.
    &

    % Communication/Writing
    Supports less experienced peers' technical skills, answering questions and
    being a resource. Documents and improves team practices.
    &

    % Data structures/Performance
    Uses advanced data structures and algorithms, such as consensus, gossip
    protocols to implement key product features.
    &

    % Security
    Understands security aspects of protocols on a deeper level, for example
    can explain differences between TLS 1.2 and 1.3 and security implications.
    Understands common attack vectors for server side or client side applications.

    \bigbreak

    Can build secure systems that will pass quality security audit that will
    uncover few to no critical system design errors.
    &

    % Networking
    Understands scalability and resilience aspects of practical implementation
    of systems leveraging networking.

    \bigbreak

    Can write fast, scalable servers and troubleshoot common networking issues
    such as connection lifecycle, pooling, backpressure and other aspects of the
    application design.
    &

    % Systems Engineering
    Not only can write data-race and deadlock free code, but implements safe
    and concurrent and/or parallel systems using minimum amount of shared state,
    granular locking - systems that are easy to read, extend and troubleshoot.

    \bigbreak

    Writes high quality design documents with few to no critical system design
    errors.
    &

    % Tooling
    \\ [13em]
  \hline
    % Level
    4
    &

    % Product Output
    \cellcolor{blue!25}
    Leads the implementation of the isolated feature/improvement that
    measurably and significantly impacts business outcomes from gathering
    requirements to getting to the market stage.
    &

    % Decision Making/Supervision
    Work is reviewed upon completion and is consistent with departmental
    objectives.
    &

    % Communication/Writing
    Writes technical articles/blog posts, delivers tech and lightning talks
    representing the company's technical vision.
    &

    % Data structures/Performance
    Leads implementation of products and standalone systems using advanced data
    structures and algorithms, such as upgrade/install distributed framework.
    &

    % Security
    Can apply production quality novel cryptographic to build security systems.
    For example, can implement strong security support for the system with eBPF
    from scratch.
    &

    % Networking
    &

    % Systems Engineering
    Can implement production grade systems leveraging advanced low-level and
    novel components of the Linux design like BPF, control groups.
    &

    % Tooling
    \\ [13em]
  \hline
    % Level
    5
    &

    % Product Output
    \cellcolor{blue!25}
    Leads the implementation of a new product line or significant part of the
    product to deliver it to the market in collaboration with all other teams.
    &

    % Decision Making/Supervision
    Focuses on providing thought leadership and works on broader organization
    projects, which requires understanding of wider business. Recognized
    internally as a subject matter expert. May direct the activities of others.
    &

    % Communication/Writing
    Writes advanced technical articles/blog posts, gaining significant industry
    traction or delivers technical talks on major conferences representing the
    company's vision.
    &

    % Data structures/Performance
    Implements customer facing features and systems using advanced data
    structures and algorithms.
    &

    % Security
    Writes technical articles on security aspects of the system, implements
    significant security product innovations in the area delivered to customers.
    &

    % Networking
    &

    % Systems Engineering
    Applies system level design to deliver new products or significant new
    components of an existing product to the market.
    &

    % Tooling
    \\ [13em]
  \hline
    % Level
    6
    &

    % Product Output
    \cellcolor{blue!25}
    Designs new data structures and algorithms solving relevant business
    problems and creating competitive advantage for the company.
    &

    % Decision Making/Supervision
    Exercises wide latitude in determining objectives and approaches to critical
    assignments.
    &

    % Communication/Writing
    Produces peer-reviewed research papers or patent applications.
    &

    % Data structures/Performance
    &

    % Security
    Researches and designs new security systems, cryptography and protocols.
    &

    % Networking
    &

    % Systems Engineering
    &

    % Tooling
    \\ [12em]
\end{tabular}

% End renewcommand wrapping.
}

\end{document}
