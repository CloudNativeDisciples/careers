\documentclass{article}

\usepackage[table]{xcolor}
\usepackage[paperwidth=800mm, paperheight=625mm]{geometry}

% Don't show page numbers.
\pagenumbering{gobble}

% Column settings for skills table.
\newcolumntype{C}{p{20em}}

\begin{document}

% Title of matrix.
\begin{center}
\begin{huge}
Site Reliability Engineering (SRE)
\end{huge}
\end{center}

\bigbreak

% Alternating row colors.
\rowcolors{2}{gray!80!gray!50}{gray!50!gray!20}

% Padding for each cell.
{\renewcommand{\arraystretch}{2}

% Skills table.
\begin{center}
\begin{tabular}{|C|C|C|C|C|C|C|C|C|}
\hline
\rowcolor{blue!60!black!10}
    \textbf{Level/Skill}
    &
    \cellcolor{blue!25}
    \textbf{Product Output}
    &
    \textbf{Decision Making/Supervision}
    &
    \textbf{Communication/Writing}
    &
    \textbf{Networking}
    &
    \textbf{Security}
    &
    \textbf{Systems Engineering}
    &
    \textbf{Site Reliability Engineering (SRE)}
    \\
\hline
    % Level
    1
    &

    % Product Output
    \cellcolor{blue!25}
    Creates a design document based on well-defined scoped requirements and
    implements it.
    &

    % Decision Making/Supervision
    Normally receives general instruction on work and new assignments.
    &

    % Communication
    Reports progress on a regular basis as required by the team's operational
    requirements. Actively solicits feedback. Participates on interview panels.
    &

    % Networking
    Understands common networking issues and troubleshooting techniques.

    \bigbreak

    Understands different networking layers (OSI model).

    \bigbreak

    Understands basic network concepts like subnets and routing. Understands
    basic network protocols like TCP/IP, ICMP, DHCP, and DNS.
    &

    % Security
    Understands operating systems security principles like iptables,
    users/groups, file permissions, and capabilities. Can apply operating system
    hardening guides like CIS Benchmarks.

    \bigbreak

    Understands basic security principles like SSH keys and TLS certificates.
    &

    % Systems Engineering
    &

    % SRE
    Understands and can make changes to existing Makefiles, shell scripts, and
    Dockerfiles.

    \bigbreak

    Understands system performance basics. Can monitor on CPU, memory, disk,
    and network utilization.

    \bigbreak

    Understands declarative configuration and can use tooling like Terraform or
    a Kubernetes operator to make changes to existing infrastructure and
    applications.
    \\ [13em]
\hline
    % Level
    2
    &

    % Product Output
    \cellcolor{blue!25}
    Can write high quality user and product focused documentation.
    &

    % Decision Making/Supervision
    Normally receives little instruction on day-to-day work, general instructions
    on new assignments.
    &

    % Communication
    Provides constructive review on peers' code and design. Helps new team
    members during their first weeks.
    &

    % Networking
    Can independently troubleshoot most network issues.

    \bigbreak

    Can setup and configure production quality network infrastructure like DNS,
    Load Balancers, and encrypted overlay networks (using IPsec or WireGuard). Can
    reason about their reliability.
    &

    % Security
    Understands and can apply basic cryptographic principles. Can configure SSH
    and TLS for a server (chooses appropriate key sizes, algorithms, and versions),
    pick strong authentication and authorization primitives, and appropriate
    encryption for data in transit and rest.

    \bigbreak

    Can setup production quality encrypted networking (like IPsec or
    WireGuard). Can reason about their reliability.
    &

    % Systems Engineering
    Understands the usage of POSIX and other APIs for Linux systems.
    &

    % SRE
    Can independently troubleshoot basic systems issues. Uses standard tools
    and logging to troubleshoot issues.

    \bigbreak

    Can use declarative languages like Terraform to build and manage infrastructure.

    \bigbreak

    Can configure alerts on latency, traffic, errors, and saturation issues.
    Uses Cloud native metrics, monitor, and alerting stacks (CloudWatch,
    Prometheus, Grafana)

    \bigbreak

    Is a member of on-call rotation and can resolve issues outlined in
    runbooks.

    \bigbreak

    Demonstrates knowledge of AWS. May have certification like AWS Certified
    SysOps Administrator.
    \\ [13em]
\hline
    % Level
    3
    &

    % Product
    \cellcolor{blue!25}
    Collaborates with the team to scope requirements, based on good
    understanding of existing longer term product vision and estimates of the
    system design of a feature of a product.
    &

    % Decision Making/Supervision
    Work is done independently and is reviewed at critical points.
    &

    % Communication
    Coordinates project deliverables alongside parallel team efforts.

    \bigbreak

    Supports less experienced peers' technical skills, answering questions and
    being a resource. Documents and improves team practices.
    &

    % Networking
    Can setup and operate a multi-region infrastructure and networking
    environment. Can reason about it's performance, reliability, and failure modes.

    \bigbreak

    Has in-depth understanding of container networking. Can write own CNI plugin
    utilizing IPSec or WireGuard on Kubernetes.

    \bigbreak

    Understands advanced networking concepts like NAT traversal and BGP.
    &

    % Security
    Can build secure systems that will pass quality security audit that will
    uncover few to no critical system design errors.

    \bigbreak

    Can apply security principals when building systems. Can utilize access
    control primitives (like IAM and RBAC) to limit access to infrastructure.
    Understands secret life cycle management in production environments. Understand
    API authentication systems, can reason about trade-offs between use of JWT,
    OIDC, and mTLS. Understands security critical events that occur within a system
    and can configure alerting on them.

    \bigbreak

    Understands advanced operating system security concepts. Can utilize
    Mandatory Access Control (MAC) systems like as SELinux or AppArmor. Understands
    container security. Can utilize control groups and namespaces to isolate and
    application.
    &

    % Systems Engineering
    Can write software (like tools and automation) that is readable/extensible
    and used in production. Understands basic testing concepts like unit and
    integration testing.
    &

    % SRE
    Excellent systems troubleshooter. Can diagnose and resolve cascading
    failures. Uses modern BPF tools for troubleshooting.

    \bigbreak

    Demonstrates advanced knowledge of Kubernetes deployments (CRDs, ingress
    controllers, Load Balancers).

    \bigbreak

    Can build and maintain reliable production CI/CD pipelines for build, test,
    and release.

    \bigbreak

    Demonstrates advanced knowledge of AWS. May have certification like AWS
    Certified Solutions Architect or Certified Kubernetes Administrator (CKA).
    \\ [13em]
\hline
    % Level
    4
    &

    % Product
    \cellcolor{blue!25}
    Leads the implementation of the isolated feature/improvement that
    measurably and significantly impacts business outcomes from gathering
    requirements to getting to the market stage.
    &

    % Decision Making/Supervision
    Work is reviewed upon completion and is consistent with departmental
    objectives.
    &

    % Communication
    Leads and clearly articulates project deliverables.

    \bigbreak

    Writes technical articles/blog posts, delivers tech and lightning talks
    representing the company's technical vision.

    \bigbreak

    Writes Root Cause Analysis (RCA) documents after incidents that help the
    team mitigate recurrence of that issue.
    &

    % Networking
    Can make changes to existing network infrastructure tooling (like load
    balancers, DNS servers, service meshes) to solve relevant business needs.
    &

    % Security
    Writes technical articles on security aspects of the system, implements
    significant security product innovations in the area delivered to customers.

    \bigbreak

    Understands and can apply advanced cryptographic principles. Understands
    hashing (including for anonymization), when to use symmetric and asymmetric
    cryptography, cipher modes, and TLS versions.

    \bigbreak

    Understands and can apply advanced network security principles. Understands
    data extrusion prevention. Understands DDoS mitigation. Understands how to
    monitor systems for rootkits and can deploy mitigation strategies when a system
    is under attack.
    &

    % Systems Engineering
    Not only can write data-race and dead- lock free code, but implements safe and
    concurrent and/or parallel systems using minimum amount of shared state,
    granular locking - systems that are easy to read, extend and troubleshoot.
    &

    % SRE
    Can build and operate large scale, stable, and reliable production platform
    environments like Kubernetes.

    \bigbreak

    Understands service availability and helps developer Service Level
    Indicators (SLI) and Service Level Objectives (SLO).

    \bigbreak

    Can deploy and operate databases at large scale. Understands index
    compaction, failover, sharding, and query performance analysis.

    \bigbreak

    Writes high quality design documents with few to no critical system design
    errors.
    \\ [13em]
\hline
    % Level
    5
    &

    % Product
    \cellcolor{blue!25}
    Leads the implementation of a new product line or significant part of the
    product to deliver it to the market in collaboration with all other teams.
    &

    % Decision Making/Supervision
    Focuses on providing thought leadership and works on broader organization
    projects, which requires understanding of wider business. Recognized
    internally as a subject matter expert. May direct the activities of others.
    &

    % Communication
    Writes advanced technical articles/blog posts, gaining significant industry
    traction or delivers technical talks on major conferences representing the
    company's vision.
    &

    % Networking
    Can create network infrastructure tooling to provide service level load
    balancing, multi-region connectivity, and observability (like Cilium).
    &

    % Security
    Researches and designs new security systems and protocols.
    &

    % Systems Engineering
    Can implement production grade systems leveraging advanced low-level and/or
    novel components like eBPF, control groups, or Noise Protocol Framework.
    &

    % SRE
    Understands and uses advanced system performance troubleshooting
    techniques like ptrace, strace, flamegraphs, or writing custom bpftrace
    programs.

    \bigbreak

    Can build advanced monitoring and anomaly detection systems.
    \\ [13em]
\hline
    % Level
    6 (internal promotion only)
    &

    % Product
    \cellcolor{blue!25}
    Designs new data structures and algorithms solving relevant business
    problems and creating competitive advantage for the company.
    &

    % Decision Making/Supervision
    Exercises wide latitude in determining objectives and approaches to critical
    assignments.
    &

    % Communication
    Produces peer-reviewed research papers or patent applications.
    &

    % Networking
    &

    % Security
    &

    % Systems Engineering
    Can design and build system for container orchestration and management like
    Kubernetes.
    &

    % SRE
    \\ [11em]
\end{tabular}
\end{center}

% End renewcommand wrapping.
}

\end{document}
