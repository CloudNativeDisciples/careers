\documentclass{article}
\usepackage{graphicx}
\usepackage{listings}
\lstset{language=C}
\graphicspath{ {images/} }
\usepackage{hyperref}
\hypersetup{
    colorlinks=true,
    linkcolor=blue,
    filecolor=magenta,      
    urlcolor=cyan,
}
\lstset{basicstyle=\sffamily}
\usepackage{array}
\usepackage[table]{xcolor}

\begin{document} 
\title{Linux Job Worker}

\section{Summary}
 
Implement a prototype job worker service that provides an API to run arbitrary Linux processes.

\section{Rationale}
 
    This exercise has two goals:

    \begin{itemize} %
      \item It helps us to understand what to expect from you as a developer, how you write production code, how you reason about API design and how you communicate when trying to understand a problem before you solve it.
      \item It helps you get a feel for what it would be like to work at Teleport, as this exercise aims to simulate our day-as-usual and expose you to the type of work we're doing here.
    \end{itemize}

    We believe this technique is not only better, but also is more fun compared to whiteboard/quiz interviews so common in the industry.  It's not without the downsides - it could take longer than traditional interviews.

    \par

    \href{https://sockpuppet.org/blog/2015/03/06/the-hiring-post/}{Some of the best teams use coding challenges.}

    We appreciate your time and are looking forward to hack on this project together.

\section{Requirements} 

  There are 6 engineering levels at Teleport. It's possible to score on level 1-5 through coding challenge.
  Level 6 is only for internal promotions. Check the engineering levels document for more details.
  To qualify for a level, implement requirements for all levels up to the one you are interested in.

  Start with a very brief Google doc that covers the edge cases and design approach and post it to the Slack channel. After the doc is approved, implement interfaces and an example program using the library.

  Add a couple of high quality tests that cover happy and unhappy scenarios.

  Split the submission into 2-3 pull requests for us to review. We will review every pull request and provide our feedback.

  We are going to compile the program, test it and get back to you.
  
  \subsection*{Level 1}

  \subsubsection*{Library}
  \begin{itemize} 
  \item[] Worker library with methods to start/stop/query status and get an output of a running job.
  \end{itemize}

  \subsubsection*{API}
  \begin{itemize} 
  \item[] Add HTTPS API to start/stop/get status of a running process. Use basic authentication.
  \end{itemize}

  \subsubsection*{Client}
  \begin{itemize} 
  \item[] Client command should be able to connect to worker service and schedule several jobs.
  \item[] Client should be able to query result of the job execution and fetch the logs.
  \end{itemize}    
    
 \subsection*{Level 2}

 \subsubsection*{API}
 \begin{itemize} 
 \item[] Use mTLS instead of basic authentication and verify client certificate. Set up strong set of cipher suites for TLS and good crypto setup for certificates.
 \end{itemize}

 \subsection*{Level 3}
    
  \subsubsection*{API}
  \begin{itemize} 
  \item[] Use \href{https://grpc.io}{GRPC} for API and add a streaming output of a running job process.
  \item[] Add streaming log output of a running job process.
  \end{itemize}

  \subsubsection*{Client}
  \begin{itemize} 
  \item[] Client should be able to stream the logs.
  \end{itemize}
  
  \subsection*{Level 4}
    
  \subsubsection*{Resource control}
  \begin{itemize}
  \item[] Add limits for amount of CPU, Memory and Disk IO using system calls.
  \end{itemize}

  \subsection*{Level 5}
    
  \subsubsection*{Isolation}
  \begin{itemize}
  \item[] Each job process should be given a separate PID, Mount and Networking namespace using system calls.
  \end{itemize}

\section{Guidance}

\subsection{Interview process}

The interview team will join the Slack channel. The team consists of the engineers who will be working with you.
Ask them about the engineering culture, work and life balance, or anything else that you would like to learn about Teleport.
  
Before writing the actual code, create a brief design document in Google Docs or markdown in Github and share with the team.

This document should consist of key trade-offs and key design approaches. Please avoid writing an overly detailed design document. Use this document to make sure the team could provide feedback on your design and demonstrate that you've investigated the problem space.

Split your code submission using pull requests and give the team an opportunity to review the PRs. A good “rule of thumb” to follow is that the final PR submission is a formality adding a small feature set - it means that the team had an opportunity to contribute the feedback during multiple well defined stages of your work.

Our team will do their best to provide a high quality review of the submitted pull requests in a reasonable time frame. You are spending your time on this, we are going to contribute our time too.

After the final submission, the interview team will assemble and vote using a "+1, -2" anonymous voting system: +1 is submitted whenever a team member accepts the submission, -2 otherwise.

In case of a positive result, we will connect you to our HR team who will collect one-two references and will work out other details. You can start the reference collection process in parallel if you would like to speed up the process.

After reference collection, our ops team will send you an offer.

In case of a negative score result, hiring manager will contact you and share a list of the key observations from the team that affected the result.

\subsection{Code and project ownership}

This is a test challenge and we have no intent of using the code you've submitted in production.
This is your work, and you are free to do whatever you feel is reasonable with it.
In the scenario when you don't pass, you can open source it with any license and use it as a portfolio project.

\subsection{Areas of focus}

Teleport focuses on networking, infrastructure and security.

These are the areas we will be evaluating in the submission:

  \begin{itemize}
  \item Use consistent coding style. We follow \href{https://github.com/golang/go/wiki/CodeReviewComments}{Golang Coding Style} for the Go language. If you are going to use a different language, please pick coding style guidelines and let us know what they are.
  \item Create one test for authentication and another for the networking component.
  \item Create one test for unhappy scenario.
  \item Make sure builds are reproducible. Pick any vendoring/packaging system that will allow us to get consistent build results.
  \item Ensure error handling and error reporting is consistent. The system should report clear errors and not crash under non-critical conditions.
  \item Avoid concurrency and networking errors. Most of the issues we've seen in production are related to data races, networking error handling or goroutine leaks. We will be looking for those errors in your code.
  \item Security. Use strong authentication and simplest, but robust authorization. Set up the strongest transport encryption you can. Test it.
  \end{itemize}

\subsection{Trade-offs}
       
Write as little code as possible, otherwise this task will consume too much time and code quality will suffer.

Please cut corners, for example configuration tends to take a lot of time, and is not important for this task.

Use hardcoded values as much as possible and simply add TODO items showing your thinking, for example:

\begin{lstlisting}[caption=TODO example]
  
  // TODO: Add configuration system.
  // Consider using CLI library to support both
  // environment variables and reasonable default values,
  // for example https://github.com/alecthomas/kingpin

\end{lstlisting}

Comments like this one are really helpful to us.
They save yourself a lot of time and demonstrate that you've spent time thinking about this problem and provide a clear path to a solution.

Consider making other reasonable trade-offs. Make sure you communicate them to the interview team.

Here are some other trade-offs that will help you to spend less time on the task:


Do not implement a system that scales or is highly performing. Describe which performance improvements you would add in the future.
High availability. It is OK if the system is not highly available. Write down how you would make the system highly available and why your system is not.
Do not try to achieve full test coverage. This will take too long. Take two key components, e.g. authentication/authorization layer and networking and implement one or two test cases that demonstrate your approach to testing.


\subsection{Pitfalls and Gotchas}
    
  To help you out, we've composed a list of things that previously resulted in a no-pass from the interview team:

    \begin{itemize}
    \item Scope creep. Candidates have tried to implement too much and ran out of time and energy. To avoid this pitfall, use the simplest solution that will work. Avoid writing too much code. We've seen candidates' code introducing caching and making many mistakes in the caching layer validation logic. Not having caching would have solved this problem.
    \item Data races. We will scan the code with a race detector and do our best to find data races in the code. Avoid global state as much as possible; if using global state, write down a good description why it is necessary and protect it against data races.
    \item Deadlocks. When using mutexes, channels or any other synchronization primitives, make sure the system won't deadlock. We've seen candidates' code holding a mutex and making a network call without timeouts in place. Be extra careful with networking and sync primitives.
    \item Unstructured code. We've seen candidates leaving commented chunks of code, having one large file with all the code, not having code structure at all.
    \item Not communicating. Some candidates have submitted all their code to the master branch without raising pull requests, which does not give us the ability to provide feedback on the various implementation phases. We are a distributed team, so structured, asynchronous communication is critical to us.
    \item Implementing custom security algorithms/authentication schemes is always a bad idea unless you are a trained security researcher/engineer. It is definitely a bad idea for this task - try to stick to industry proven security methods as much as possible.
    \end{itemize}

\subsection{Scoring}
    
We want to be as transparent as possible on how we will be scoring your submission.
The following table provides a description of different areas you will be evaluated on and how they will affect your overall score.

\begin{center}
\rowcolors{2}{gray!80!gray!50}{gray!50!gray!20}
\begin{tabular}{ | m{25em} | m{5em}| m{5em} | }
  \hline
  \rowcolor{blue!60!black!10}
  Description & Points Awarded & Points Subtracted \\
  \hline
  The submitted code has a clear and modular structure & +1 & -1 \\
  \hline   
  The candidate communicated their progress during the interview & +1 & -1 \\
  \hline
  The program builds are reproducible & +1 & -1 \\
  \hline
  README provides clear instructions & +1 & -1 \\
  \hline
  The candidate outlined the key design points in the design document & +1 & -1 \\
  \hline
  The code has no obvious data races and deadlocks & +1 & -1 \\
  \hline
  The code provides examples of tests covering key components & +1 & -1 \\
  \hline
  The code provides clear error handling and reporting & +1 & -1 \\
  \hline
  The program is working according to the specification & +1 & -1 \\
  \hline
  The candidate demonstrates ability to handle and apply feedback & +1 & -1 \\
  \hline
  The client-server communication is implemented in a secure way & +1 & -1 \\
  \hline
\end{tabular}
\end{center}

\subsection{Scoring}

It is OK to ask the interview team questions. Some folks stay away from
asking questions to avoid appearing less experienced, so we provide examples of questions
to ask and questions we expect candidates to figure out on their own.

Here is a great question to ask:

``Is it OK to pre-generate secret data and put the secrets in the repository for a proof of concept? I will add a note that we will auto-generate secrets in the future.''

It demonstrates that you thought about this problem domain, recognize the trade off and are saving you and the team time by not implementing it.

This is the question we expect candidates to figure out on their own:

``What version of Go should I use? What dependency manager should I use?''

Unless specified in the requirements, pick the solution that works best for you.

\section{Tools}

This task should be implemented in Go, C++ or Rust and should work on 64-bit Linux machines.

\section{Timing}

It should take you from 4 to 24 full hours to complete the challenge.

You can split coding over a couple of weekdays or weekends and find time to ask questions and receive feedback.

Once you join the Slack channel, you have a maximum of 2 weeks to complete the challenge.

Within this timeframe, we don't give higher scores to challenges submitted more quickly.
We only evaluate the quality of the submission.

We only start the coding challenge if there are several open positions available and let
all candidates finish the code submission.

\end{document}
